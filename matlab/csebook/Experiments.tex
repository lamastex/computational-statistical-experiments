\chapter{Inference for Statistical Experiments}\label{S:StatExps}
\section{Introduction}\label{S:ExpsIntro}
We formalize the notion of a staistical experiment.  Let us first motivate the need for a statistical experiment.  Recall that statistical inference or learning is the process of using observations or data to infer the distribution that generated it.  A generic question is:
\[
\text{Given realizations from $X_1, X_2, \ldots, X_n \sim$ some unknown DF $F$, how do we infer $F$} ?
\]
However, to make this question tractable or even sensible it is best to restrict ourselves to a particular  class or family of DFs that may be assumed to contain the unknown DF $F$.

\begin{definition}[Experiment]
A statistical experiment $\EE{E}$ is a set of probability distributions (DFs, PDFs or PMFs) 
$\Pz := \{\P_{\theta} : \theta \in \BB{\Theta} \}$ associated with a RV $X$ and indexed by the set $\BB{\Theta}$.  
We refer to $\BB{\Theta}$ as the parameter space or the index set and $d:\BB{\Theta} \rightarrow \Pz$ that associates to each $\theta \in \BB{\Theta}$ a probability $\P_{\theta} \in \Pz$ as the index map:
\[ \BB{\Theta} \ni \theta \mapsto \P_{\theta} \in \Pz \enspace .\]
\end{definition}

\section{Some Common Experiments}
Next, let's formally consider some experiments we have already encountered.
\begin{Exp}[The Fundamental Experiment]\label{Exp:Uniform01}
The `uniformly pick a number in the interval $[0,1]$' experiment is the following singleton family of DFs  :
\[
\Pz = \{ \,  F(x) = x \BB{1}_{[0,1]}(x)  \, \} 
\]
where, the only distribution $F(x)$ in the family $\Pz$ is a re-expression of~\eqref{E:Uniform01DF} using the indicator function $\BB{1}_{[0,1]}(x)$.  The parameter space of the fundamental experiment is a singleton whose DF is its own inverse, ie.~$F(x) = F^{[-1]}(x)$. 
Recall from Exercise~\ref{underMPSA} that this is equivalent to infinitely many independent and identical $\bernoulli(1/2)$ trials, i.e., independently tossing a fair coint infinitely many times.
%The two dimensional parameter space or index set for this experiment is $\BB{\Theta} = \{ -\infty < a < b < \infty \} = \{ (a,b) \in \Rz \times \Rz :  a < b \}$, a half-plane.
\end{Exp}

\begin{Exp}[Bernoulli]\label{Exp:Bernoulli}
The `toss 1 times' experiment is the following family of densities (PMFs) :
\[
\Pz = \{ \,  f(x; \theta) :  \theta \in [0,1] \, \} 
\]
where, $f(x; \theta)$ is given in~\eqref{E:Bernoullipdf}.  The one dimensional parameter space or index set for this experiment is  $\BB{\Theta} = [0,1] \subset \Rz$.
\end{Exp}

\begin{Exp}[Point~Mass]\label{Exp:PointMass}
The `deterministically choose a specific real number' experiment is the following family of DFs :
\[
\Pz = \{ \,  F(x; a) :  a \in \Rz \, \} 
\]
where, $F(x; a)$ is given in~\eqref{E:PointMasscdf}.  The one dimensional parameter space or index set for this experiment is $\BB{\Theta} = \Rz$, the entire real line.
\end{Exp}
Note that we can use the PDF's or the DF's to specify the family $\Pz$ of an experiment.  When an experiment can be parametrized by finitely many parameters it is said to a {\bf parametric} experiment.  \hyperref[Exp:Bernoulli]{Experiment~\ref*{Exp:Bernoulli}} involving discrete RVs as well as \hyperref[Exp:PointMass]{Experiment \ref*{Exp:PointMass}} are {\bf parametric} since they both have only one parameter (the parameter space is one dimensional for Experiments \ref*{Exp:Bernoulli} and \ref*{Exp:PointMass}).   The \hyperref[Uniform01]{Fundamental Experiment \ref*{Exp:Uniform01}} involving the continuous RV of \hyperref[M:Uniform01]{Model \ref*{M:Uniform01}} is also parametric since its parameter space, being a point, is zero-dimensional.  The next example is also parametric and involves $(k-1)$-dimensional families of discrete RVs.

\begin{Exp}[{de~Moivre[k]}]\label{Exp:GenDiscrete}
The `pick a number from the set $[k] := \{1,2,\ldots,k\}$ somehow' experiment is the following family of densities (PMFs) :
\[
\Pz = \{ \,  f(x; \theta_1,\theta_2,\ldots,\theta_k) :   (\theta_1,\theta_2,\ldots,\theta_k) \in \bigtriangleup_k \, \} 
\]
where, $f(x; \theta_1,\theta_2,\ldots,\theta_k)$ is any PMF such that 
\[
f(x; \theta_1,\theta_2,\ldots,\theta_k) = \theta_x, \qquad x \in \{1,2,\ldots,k\} \ .
\]
The $k-1$ dimensional parameter space $\BB{\Theta}$ is the $k$-Simplex $\bigtriangleup_k$.  This as an `exhaustive' experiment since all possible densities over the finite set $[k] := \{1,2,\ldots,k\}$ are being considered that can be thought of as ``the outcome of rolling a convex polyhedral die with $k$ faces and an arbirtary center of mass specified by the $\theta_i$'s.''
\begin{figure}
\caption{Geometry of the $\BB{\Theta}$'s for $\demoivre[k]$ Experiments with $k \in \{1, 2, 3, 4\}$.}
\vspace{5cm}
\end{figure}
\end{Exp}
An experiment with infinite dimensional parameter space $\BB{\Theta}$ is said to be {\bf nonparametric} .  Next we consider two nonparametric experiments.
\begin{Exp}[All DFs]\label{Exp:AllDFs}
The `pick a number from the Real line in an arbitrary way' experiment is the following family of distribution functions (DFs) :
\[
\Pz = \{ \,  F(x; F) :  F~is~a~DF \, \} = \BB{\Theta} 
\]
where, the DF $F(x; F)$ is indexed or parameterized by itself. Thus, the parameter space 
\[
\BB{\Theta}=\Pz=\{ \text{all DFs}\}
\]
is the infinite dimensional space of {\bf All DFs} ''.
\end{Exp}
Next we consider a {\bf nonparametric} experiment involving continuous RVs.
\begin{Exp}[Sobolev Densities]\label{Exp:Sob}
The `pick a number from the Real line in some reasonable way' experiment is the following family of densities (pdfs) :
\[
\Pz = \left\{ \,  f(x; f) :  \int(f''(x))^2 < \infty \, \right\} = \BB{\Theta} 
\]
where, the density $f(x; f)$ is indexed by itself.  Thus, the parameter space $\BB{\Theta}=\Pz$ is the infinite dimensional {\bf Sobolev space} of ``not too wiggly functions''.
\end{Exp}

\section{Typical Decision Problems with Experiments}
Some of the concrete problems involving experiments include:
\begin{itemize}
\item {\bf Simulation:} Often it is necessary to simulate a RV with some specific distribution to gain insight into its features or simulate whole systems such as the air-traffic queues at `London Heathrow' to make better management decisions.
\item {\bf Estimation:} 
\begin{enumerate}
\item {\bf Parametric Estimation:} Using samples from some unknown DF $F$ parameterized by some unknown $\theta$, we can estimate $\theta$ from a statistic $T_n$ called the estimator of $\theta$ using one of several methods (maximum likelihood, moment estimation, or parametric bootstrap).
\item {\bf Nonparametric Estimation of the DF:}  Based on $n$ IID observations from an unknown DF $F$, we can estimate it under the general assumption that $F \in \{ \text{all DFs} \}$.
\item {\bf Confidence Sets:}  We can obtain a $1-\alpha$ confidence set for the point estimates, of the unknown parameter $\theta \in \BB{\Theta}$ or the unknown  DF $F \in \{ \text{all DFs} \}$
\end{enumerate}
\item {\bf Hypothesis Testing:}  Based on observations from some DF $F$ that is hypothesized to belong to a subset $\BB{\Theta}_0$ of $\BB{\Theta}$ called the space of null hypotheses, we will learn to test (attempt to reject) the falsifiable null hypothesis that $F \in \BB{\Theta}_0 \subset \BB{\Theta}$.
\item $\ldots $ 
\end{itemize}


\section{Decision Problems and Procedures for Actions}\label{S:Decisions}

Write down the Table from lectures 1 \& 2 giving examples of decision problems, procedures and action spaces for typical estimation, hypothesis testing and prediction problems with associated principles (Maximum Likelihood, Empirical Risk Minimisation where risk is expectation of specific loss functions, etc.) and algorithms (including optimisation (Stochastic)Newton/gradient-descent, etc.).

\vspace{10cm}~\\


