\section{Exercises in Conditional Probability}\label{S:xsCondProb}

\begin{ExerciseList}

\Exercise
What gives the greater probability of hitting some target at least
  once: 
\begin{enumerate}
\item  hitting in a shot with probability $\frac{1}{2}$ and firing 1 shot, or 
\item hitting in a shot with probability $\frac{1}{3}$ and firing 2 shots?
\end{enumerate}
First guess. Then calculate.

\Answer
We can assume  that the first shot is independent of
the second shot so we can multiply the probabilities here.

For case A, there is only one shot so the probability of hitting at least once is $\frac{1}{2}$.

For case B, the probability of missing both shots is
$\frac{2}{3}\,\frac{2}{3}\,=\,\frac{4}{9}$,  so  the probability hitting some target at least once is
  \[ 1- \p(\text{missing the target both times})\;=\; 1-\frac{4}{9}\;=\;\frac{5}{9}\]
Therefore, case B has the greater probability of hitting the target at least once.

\Exercise
Suppose we independently roll two fair dice each of whose faces are marked by numbers
  $\sf 1$,$\sf 2$,$\sf 3$,$\sf 4$, $\sf 5$ and $\sf 6$.
\begin{enumerate}
\item List the sample space for the experiment if we note the   numbers
  on the 2 upturned faces.
\item What is the probability of obtaining a sum greater than 4 but less
  than 7?
\end{enumerate}

\Answer
\be
\item The sample space is
$$
\begin{aligned}
\{
\mathsf{(1,1),(1,2),(1,3),(1,4),(1,5),(1,6),\quad (2,1),(2,2),(2,3),(2,4),(2,5),(2,6),}\\
\mathsf{(3,1),(3,2),(3,3),(3,4),(3,5),(3,6),\quad (4,1),(4,2),(4,3),(4,4),(4,5),(4,6),}\\
\mathsf{(5,1),(5,2),(5,3),(5,4),(5,5),(5,6),\quad (6,1),(6,2),(6,3),(6,4),(6,5),(6,6)}
\}
\end{aligned}
$$
Note: Order matters here. For example, the outcome ``16'' refers to a ``1'' on the first die and a ``6'' on the second,
whereas the outcome ``61'' refers to a ``6'' on the first die and a ``1'' on the
second.

\item First tabulate all possible sums as follows:
$$
\begin{array}{c|cccccc}
+       &       1       &       2       &       3       &       4       &       5       &       6       \\\hline
1       &       2       &       3       &       4       &       \mathbf{5}      &       \mathbf{6}      &       7       \\
2       &       3       &       4       &       \mathbf{5}      &       \mathbf{6}      &       7       &       8       \\
3       &       4       &       \mathbf{5}      &       \mathbf{6}      &       7       &       8       &       9       \\
4       &       \mathbf{5}      &       \mathbf{6}      &       7       &       8       &       9       &       10      \\
5       &       \mathbf{6}      &       7       &       8       &       9       &       10      &       11      \\
6       &       7       &       8       &       9       &       10      &       11      &       12      \\
\end{array}$$
 Let $A$ be the event {\em the sum is 5} and $B$ be the event {\em
    the sum is 6}, then $A$ and $B$ are mutually exclusive events with
  probabilities \[\p(A) = \frac{4}{36} \quad \text{and}\quad \p(B) =
  \frac{5}{36} \,.\]
Therefore,
   $$\p(4<\textrm{sum }<7)\;=\; \p( A \cup B) \;=\; \p(A)\,+\,\p(B)\;=\; \frac{4}{36} +
\frac{5}{36} \;=\;\frac{1}{4}$$
\ee


\Exercise
Based on past experience, 70\% of students in a certain course pass the midterm test.  
The final exam is passed by 80\% of those who passed the midterm test, but only by 40\% of those who fail the midterm test.  
What fraction of students pass the final exam?

\Answer
First draw a tree  with the first split based on the outcome of the midterm test and the second on the outcome of the final exam.  
Note that the probabilities involved in this second branch are {\em conditional} probabilities that depend on the outcome of the midterm test.  
Let $A$ be the event that the student passes the final exam and let $B$ be the event that the student passes the midterm test.

\begin{center}
\begin{picture}(100,160)(75,30){\drawline(0,90)(75,120)(125,140)
\drawline(75,120)(125,100)\drawline(0,90)(75,60)(125,40)
\drawline(75,60)(125,80)
\put(55,125){$B$}\put(55,50){$B^c$}
\put(130,140){$A$\quad $\p(A \cap  B) = 0.56$}\put(130,100){$A^c$ \quad
  $\p(A^c\cap B)= 0.14$}
\put(130,80){$A$ \quad $\p(A\cap B^c) = 0.12$ }\put(130,40){$A^c$ \quad
  $\p(A^c\cap B^c)= 0.18$}
\put(100,137){0.8}\put(100,112){0.2}
\put(100,77){0.4}\put(100,52){0.6}
\put(30,110){0.7}\put(30,85){0.3}}
\put(40,165){Midterm}
\put(120,165){Final}
\end{picture}
\end{center}

Then the probability of passing the final exam is:
\[ \p(A)\;=\; 0.56\,+\,0.12\;=\;0.68\,.\]

To do this with formulae,  partitioning according to the midterm test
result  and using the multiplication rule, we get:
\begin{eqnarray*}
\p(A)&=& \p(A\cap B)+\p(A\cap B^c)\\
&=& \p(A|B)\p(B)+\p(A|B^c)\p(B^c)\\
&=& (0.8)(0.7)+(0.4)(0.3)\;=\;0.68
\end{eqnarray*}

\Exercise
A small brewery has two bottling machines.  
Machine 1 produces 75\% of the bottles and machine 2 produces 25\%.  
One out of every 20 bottles filled by machine 1 is rejected for some reason, while one out of every 30 bottles filled by machine 2 is rejected. 
What is the probability that a randomly selected bottle comes from machine 1 given that it is accepted?
\Answer
Let $A$ be the event that bottles are produced by machine 1; and $A^c$ is the event that bottles are produced by machine 2. 
$R$ denotes the event that the bottles are rejected; and $R^c$ denotes the event that the bottles are accepted.  
We know the following probabilities:

\[ \p(A) \;=\; 0.75 \quad \text{and} \quad \p(A^c) \;=\; 0.25\]
\[ \p(R | A) \;=\; \frac{1}{20} \quad \text{and } \quad  \p(R^c | A) \;=\;
\frac{19}{20}\]

\[ \p(R | A^c) \;=\; \frac{1}{30} \quad \text{and} \quad  \p(R^c | A^c) \;=\;
\frac{29}{30}\]

We want $\p(A|R^c)$ which is give by
$$\p(A|R^c)\;=\;\frac{\p(R^c\cap A)}{\p(R^c)}\;=\;\frac{\p(R^c\cap A)}{\p(R^c\cap A)+\p(R^c\cap A^c)}$$
where, $$\p(R^c\cap A)\;=\;\p(R^c| A)\p(A)\;=\;\frac{19}{20}\times0.75$$
and, $$\p(R^c\cap A^c)\;=\;\p(R^c|A^c)\p(A^c)\;=\;\frac{29}{30}\times0.25$$
Therefore,
$$\p(A|R^c)\;=\;\frac{\frac{19}{20}\times0.75}{\frac{19}{20}\times
  0.75+\frac{29}{30}\times0.25}\;\approxeq \;0.747\quad $$\\[6pt]

The  tree diagram for this problem is:
\begin{center}
\begin{picture}(100,140)(75,30){\drawline(0,90)(75,120)(125,140)
\drawline(75,120)(125,100)\drawline(0,90)(75,60)(125,40)
\drawline(75,60)(125,80)
\put(60,120){$A$}
\put(60,50){$A^c$}
\put(130,140){$\mathsf{R}$}
\put(130,100){$\mathsf{R^c}$ \;\;$\p(A\cap R^c)= 0.75\times 19/20$}
\put(130,80){$\mathsf{R}$}
\put(130,40){$\mathsf{R^c}$ \;\;$\p(A^c\cap R^c)= 0.25\times 29/30$}
\put(90,140){$1/20$}\put(90,95){$19/20$}
\put(90,75){$1/30$}\put(90,35){$29/30$}
\put(30,110){$0.75$}\put(30,60){$0.25$}
\put(40,160){Machine 1}
\put(120,160){Rejection}
}
\end{picture}
\end{center}

So the required probability is 
$$\p(A|R^c)\;=\;\frac{\p(R^c\cap A)}{\p(R^c)}\;=\;\frac{\frac{19}{20}\times0.75}{\frac{19}{20}\times
  0.75+\frac{29}{30}\times0.25}\;\approxeq \;0.747$$

\Exercise
A process producing micro-chips, produces 5\% defective, at random.  
Each micro-chip is tested, and the test will correctly detect a defective one $4/5$ of the time, and if a good micro-chip is tested the test will declare it is defective with probability $1/10$.
\begin{itemize}
 \item[(a)]If a micro-chip is chosen at random, and tested to be good, what was the probability that it was defective anyway?
\item[(b)]If a micro-chip is chosen at random, and tested to be defective, what was the probability that it was good anyway?
\item[(c)]If 2 micro-chips are tested and determined to be good, what is
  the probability that at least one is in fact defective?
\end{itemize}
\Answer
Let  the event that a micro-chip is defective be $D$, and the event  that the test  is correct be  $C$.  
So  the probability that the micro-chip is defective is $P(D)=0.05$, and the probability that it is effective is $P(D^c)=0.95$.

The probability that  the test correctly detects a defective micro-chip is the conditional probability $P(C|D)=0.8$, and the probability that if a good micro-chip is tested but the test declares it is defective is the conditional probability $P(C^c|D^c)=0.1$.  
Therefore, we also have the probabilities $P(C^c|D)=0.2$, and $P(C|D^c)=0.9$.

Moreover, the probability that a micro-chip is defective, and has been declared as defective is $$P(C\cap D)\;=\;P(C|D)P(D)\;=\;0.8\times0.05\;=\;0.04\,.$$
The probability that a micro-chip is effective, and has been declared as effective is $$P(C\cap D^c)\;=\;P(C|D^c)P(D^c)\;=\;0.9\times0.95\;=\;0.855\,.$$
The probability that a micro-chip is defective, and has been declared as effective is $$P(C^c\cap D)\;=\;P(C^c|D)P(D)\;=\;0.2\times 0.05=0.01\,.$$
The probability that a micro-chip is effective, and has been declared as defective is $$P(C^c\cap D^c)\;=\;P(C^c|D^c)P(D^c)\;=\;0.1\times0.95\;=\;0.095\,.$$

The tree diagram for  these events  and probabilities is:
\begin{center}
\begin{picture}(100,140)(75,30){\drawline(0,90)(75,120)(125,140)
\drawline(75,120)(125,100)\drawline(0,90)(75,60)(125,40)
\drawline(75,60)(125,80)
\put(60,120){$D$}
\put(60,50){$D^c$}
\put(130,140){$C$\;\;$P(C\cap D)= 0.04$}
\put(130,100){$C^c$ \;\;$P(C^c\cap D)=  0.01$}
\put(130,80){$C$\;\;$P(C\cap D^c)= 0.855$}
\put(130,40){$C^c$ \;\;$P(C^c\cap D^c)= 0.095$}
\put(90,140){$0.8$}\put(90,95){$0.2$}
\put(90,75){$0.9$}\put(90,35){$0.1$}
\put(30,110){$0.05$}\put(30,60){$0.95$}
\put(40,160){ Defective}
\put(100,160){ Tested Correctly}
}
\end{picture}
\end{center}

\begin{itemize}
\item[(a)]
If a micro-chip is tested to be good, it could be defective but tested incorrectly, or it could be effective and tested correctly. Therefore, the probability that the micro-chip is tested good, but it is actually defective is
$$\frac{P(C^c\cap D)}{P(C^c\cap D)+P(C\cap D^c)}\;=\;\frac{0.01}{0.01+0.855}\;\approxeq\; 0.012$$
\item[(b)] 
Similarly,  the probability that a micro-chip is  tested to be defective, but it was good is
$$\frac{P(C^c\cap D^c)}{P(C\cap D)\;+\;P(C^c\cap
  D^c)}=\frac{0.095}{0.095+0.04} \approxeq 0.704$$
\item[(c)]
The  probability that  both  the micro-chips are effective, and have been
tested and determined to be good,  is
\[\left(\frac{P(C\cap D^c)}{P(C^c\cap D)+P(C\cap D^c)}\right)^2\]
and so  the probability that at least one is defective is:
$$1\;-\;\left(\frac{P(C\cap D^c)}{P(C^c\cap D)+P(C\cap
    D^c)}\right)^2\;=\;1-\left(\frac{0.855}{0.01+0.855 }\right)^2\;\approxeq\;0.023$$
\end{itemize}


\Exercise
Suppose that $\frac{2}{3}$  of all gales are  force 1, $\frac{1}{4}$  are force 2 and $\frac{1}{12}$  are force 3.  
Furthermore, the probability that force 1 gales cause damage is
$\frac{1}{4}$,  the probability that force 2 gales cause damage is
$\frac{2}{3}$  and  the probability that force 3 gales cause damage is
$\frac{5}{6}$.

\begin{itemize}
\item[(a)]If a  gale is reported,  what is the probability of it causing damage?
\item[(b)] If the gale caused damage, find  the probabilities that it
  was  of:  force 1;  force 2;  force 3.
\item[(c)] If the gale did  NOT cause damage, find  the probabilities
  that it was of: force 1; force 2; force 3.
\end{itemize}
\Answer
\begin{itemize}
\item[(a)] Let $F1$ be the event a gale of force 1 occurs, let $F2$
be the event a gale of force 2 occurs and $F3$ be the event a gale
of force 3 occurs.  Now we know that
\[ P(F1)\; =\; \frac{2}{3}, \quad  P(F2) \;= \;\frac{1}{4},  \quad P(F3) \;= \;\frac{1}{12}\,.\]
If $D$ is the event that a gale causes damage, then we also know the
following conditional probabilities:
\[ P(D|F1) \;= \;\frac{1}{4}, \quad  P(D|F2) \;=\; \frac{2}{3},  \quad  P(D|F3)\; = \;\frac{5}{6}\,.\]
The probability that a reported gale causes damage is
\[P(D)\;=\;P(D\cap F1)\;+\;P(D\cap F2)\;+\;P(D\cap F3)\]
where
\[P(D\cap F1)\;=\; P(D| F1)P(F1)\;=\;\frac{1}{4} \times \frac{2}{3}
\;=\; \frac{1}{6}\,, \]
\[P(D\cap F2)\;=\; P(D| F2)P(F2)\;=\;\frac{2}{3} \times \frac{1}{4}
\;=\; \frac{1}{6}\,, \]
and
\[P(D\cap F3)\;=\; P(D| F3)P(F3)\;=\;\frac{5}{6} \times \frac{1}{12}
\;=\; \frac{5}{72} \,.\]
Hence \[P(D)\;=\;  \frac{1}{6}\,+\,  \frac{1}{6}\,+\,\frac{5}{72}\;=\; \frac{29}{72} \]

\item[(b)] Knowing that the gale did cause damage we can calculate the probabilities that it was of the various forces using the probabilites in (a) as follows (Note: $P(D\cap F1)= P(F1\cap D)$ etc.):
{$$P(F1|D)\;=\;\frac{P(F1\cap D)}{P(D)}\;=\;\frac{1/6}{29/72}\;=\;\frac{12}{29}$$
$$P(F2|D)\;=\;\frac{P(F2\cap D)}{P(D)}\;=\;\frac{1/6}{29/72}\;=\;\frac{12}{29}$$
$$P(F3|D)\;=\;\frac{P(F3\cap D)}{P(D)}\;=\;\frac{5/72}{29/72}\;=\;\frac{5}{29}$$
}

\item[(c)]First note that the probability that a reported gale does NOT
  cause damage is:
\[ P(D^c)\;=\; 1-P(D)\;=\; 1- \frac{29}{72}\;=\; \frac{43}{72}\,.\]
Now we need to find probabilities like $P(F1\cap D^c)$. The best way to
do this is to use the partitioning idea of the  ``Total Probability
Theorem'', and write:
\[ P(F1) \;=\;  P(F1\cap D^c) \;+\; P(F1\cap D)\,,\]
Rearranging this gives
\[  P(F1\cap D^c) \;= \;P(F1) \;- \; P(F1\cap D) \]
and so
\[P(F1|D^c)\;=\;\frac{P(F1\cap D^c)}{P(D^c)}\;=\;\frac{P(F1)-P(D\cap
  F1)}{P(D^c)}\;=\;\frac{2/3-1/6}{43/72}\;=\;\frac{36}{43}\,.
\]
Similarly,
\[P(F2|D^c)\;=\;\frac{P(F2\cap D^c)}{P(D^c)}\;=\;\frac{P(F2)-P(D\cap F2)}{P(D^c)}\;=\;\frac{1/4-1/6}{43/72}\;=\;\frac{6}{43}\,,\]
and
\[P(F3|D^c)\;=\;\frac{P(F3\cap D^c)}{P(D^c)}\;=\;\frac{P(F3)-P(D\cap F3)}{P(D^c)}\;=\;\frac{1/12-5/72}{43/72}\;=\;\frac{1}{43}\,.\]

\end{itemize}
\newpage


\Exercise
{**}The sensitivity and specificity of a medical diagnostic test for a disease are defined as follows:
\begin{eqnarray*}
\text{sensitivity } &=& \p \left( \text{ test is positive } | \text{ patient has the disease } \right) \enspace ,\\
\text{specificity} &=& \p \left( \text{ test is negative } | \text{ patient does not have the disease } \right) \enspace .
\end{eqnarray*}
Suppose that a medical test has a sensitivity of $0.7$ and a specificity of $0.95$.  
If the prevalence of the disease in the general populaiton is $1\%$, find
\begin{itemize}
\item[(a)] the probability that a patient who tests positive actually has the disease,
\item[(b)] the probability that a patient who tests negative is free from the disease.
\end{itemize}

\Exercise
{**}The detection rate and false alarm rate of an intrusion sensor are defined as
\begin{eqnarray*}
\text{detection rate} &=& \p \left( \text{ detection declared } | \text{ intrusion } \right) \enspace ,\\
\text{false alarm rate} &=& \p \left( \text{ detection declared } | \text{ no intrusion } \right) \enspace .
\end{eqnarray*}
If the detection rate is $0.999$ and the false alarm rate is $0.001$, and the probability of an intrusion occuring is $0.01$, find
\begin{itemize}
\item[(a)] the probability that there is an intrusion when a detection is declared,
\item[(b)] the probability that there is no intrusion when no detection is declared.
\end{itemize}

\Exercise
{**}Let $A$ and $B$ be events such that $\p(A) \neq 0$ and $\p(B) \neq 0$.  
When $A$ and $B$ are disjoint, are they also independent?  
Explain clearly why or why not.

\end{ExerciseList}


