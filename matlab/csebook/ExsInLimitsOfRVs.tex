\section{Exercises in Limit Laws of Statistics}\label{S:xsLimitLaws}% {S:xsCFs}% {S:xsExpectationsOfRVs} %S:xsMultivariateRVs

\begin{ExerciseList}

\Exercise
Suppose you plan to obtain a simple random sequence (SRS) --- also known as independent and identically distributed (IID) sequence --- of $n$ measurements from an instrument.  
This instrument has been calibrated so that the distribution of measurements made with it have population variance of $1/4$.  
Your boss wants you to make a point estimate of the unknown population mean from a SRS of sample size $n$.  
He also insists that the tolerance for error has to be $1/10$ and the probability of meeting this tolerance should be just above $95\%$.  
Use CLT to find how large should $n$ be to meet the specifications of your boss.
\Answer
~\\
%Using the CLT implied Equation (61) in the lecture notes and further noting from the Standard Normal Table that when $z_{\alpha/2}=2$ we get the desired $1-\alpha=0.9772>0.95$ we get
%\[
%n = \left( \sqrt{(V(X_1)} z_{\alpha/2}) / \epsilon \right)^2 = \left( (\sqrt{1/4}\times 2) / (1/10) \right)^2
%= \left( ((1/2)\times 2) / (1/10) \right)^2 = 10^2 = 100
%\]
We want $1-\alpha = 0.95$, and from the standard Normal Table we know that the corresponding $z_{\alpha/2}=1.96$.   
Then we can get the right sample size $n$ from the CLT implied Equation (61) in the lecture notes, which is,
\[
n = \left( \sqrt{(V(X_1)} z_{\alpha/2}) / \epsilon \right)^2 \enspace ,
\] 
as follows:
\begin{eqnarray*}
n &= \left( \sqrt{(V(X_1)} z_{\alpha/2}) / \epsilon \right)^2 = \left( (\sqrt{1/4}\times 1.96) / (1/10) \right)^2\\
&= \left( ((1/2)\times 1.96) / (1/10) \right)^2 = (0.98 \times 10)^2 = 9.8^2 = 96.04
\end{eqnarray*}
Finally, by rounding $96.04$ up to the next largest integer we need $n=97$ measurements to meet the specifications of your boss (at least up to the approximation provided by the CLT).


\Exercise 
Suppose the collection of RVs $X_1,X_2, \ldots, X_n$ model the number of errors in $n$ computer programs named $1,2,\ldots,n$, respectively.  Suppose that the RV $X_i$ modeling the number of errors in the $i$-th program is the $\textrm{Poisson}(\lambda=5)$ for any $i=1,2,\ldots,n$.  Further suppose that they are independently distributed.  Succinctly, we suppose that 
\[
X_1,X_2,\ldots,X_n \overset{\IID}{\sim} \poisson(\lambda=5) \ . 
\]
Suppose we have $n=125$ programs and want to make a probability statement about $\overline{X}_{125}$ which is the average error per program out of these $125$ programs.  Since $E(X_i) = \lambda=5$ and $V(X_i)=\lambda=5$, we want to know how often our sample mean $\overline{X}_{125}$ differs from the expectation of $5$ errors per program.  
Using the CLT find the $P(\overline{X}_{125} < 5.5)$.
\Answer
~\\
By CLT, $\frac{\sqrt{n}(\overline{X}_n - \e(X_1))}{\sqrt{\V(X_1)}} \rightsquigarrow Z \sim \mathrm{Normal}(0,1)$.  
So we need to apply the ``standardization'' to both sides of the inequality that is defining the event of interest:
$$\{\overline{X}_n < 5.5\} \enspace ,$$ 
in order to find its probability $\p(\overline{X}_n < 5.5)$.
~\\
\begin{eqnarray}
\p(\overline{X}_n < 5.5) 
&=& P \left( \frac{\sqrt{n}(\overline{X}_n - \e(X_1))}{\sqrt{\V(X_1)}} < \frac{\sqrt{n}(5.5-\e(X_1))}{\sqrt{\V(X_1)}} \right) \notag \\
&\approxeq& P \left( Z < \frac{\sqrt{n}(5.5-\lambda)}{\sqrt{\lambda}} \right) \qquad \text{{\scriptsize [
since we know/assume that $\e(X_1)=\V(X_1)=\lambda$]}} \notag \\
&=& P \left( Z < \frac{\sqrt{125}(5.5-5)}{\sqrt{5}} \right) \qquad \text{{\scriptsize [Since, $\lambda=5$ and $n=125$ in this Example]}} \notag \\
&=& \p(Z \leq 2.5) = \Phi(2.5) = 0.9938\ . \notag
\end{eqnarray}
(source: Wasserman, {\em All of Statistics}, Springer, p.~78, 2003)


\Exercise
What is the distribution of $\sum_{i=1}^n{X_i}/n$ as $n \to \infty$ when $X_i \overset{IID}{\sim} \uniform(-10,10)$?
\Answer
HINT: Use the LLN after finding the population mean of $X_i$.
\Exercise
What is the distribution of $\sum_{i=1}^n{X_i}/\sqrt{n}$ as $n \to \infty$ when $X_i \overset{IID}{\sim} \uniform(-10,10)$?
\Answer
HINT: Use the CLT after finding the population mean and variance of $X_i$.
\end{ExerciseList}
