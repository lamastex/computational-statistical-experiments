\section{Exercises in Multivariate Random Variables}\label{S:xsMultivariateRVs}
\begin{ExerciseList}
%question 9 (cts )
\Exercise
Find the probability that none of the three bulbs in a traffic signal, that are assumed to have independent life-times (i.e., the time during which they are operational), need to be replaced during the first 1200 hours of operation if the length of time before a single bulb needs to  be replaced is a continuous random variable $X$ with density
$$f(x)\;=\;\begin{cases}6\left(0.25-(x-1.5)^2\right)&1<x<2\\0&\textrm{otherwise}\end{cases}\enspace.$$
Note: $X$ is measured in multiples of 1000 hours.
\Answer
The probability that  \emph{one}  light bulb doesn't need to be replaced in  1200 hours is:
\ba{\P(X>1.2)&\;=\;1-\P(X<1.2)\\[3pt]
&=\;1\;-\;\int^{1.2}_1 6 (0.25-(x-1.5)^2)\,dx\\[3pt]
&=\;1\;-\;\int^{1.2}_1 6(0.25-x^2+3x-2.25)\,dx\\[3pt]
&=\;1\;-\;\int^{1.2}_1 (-6x^2+18x-12)\,dx\\[3pt]
&=\;1\;-\;\left[-2x^3+9x^2-12x\right]^{1.2}_1\\[3pt]
&=\;1-0.1040\\[3pt]
&\;=\;0.8960}
Assuming  that   the three light bulbs function  independently of  each
other, the probability that none of them need to be replaced in the
first 1200 hours is
$$\P(\{X_1>1.2\}\cap\{X_2>1.2\}\cap\{X_3>1.2\}\;=\;0.8960^3\;=\;
0.7193$$
where $X_i$ is the length of time that bulb $i$ lasts.

\Exercise
Let $(X,Y)$ be a continuous \rv~with joint probability density function (JPDF)
\[
f_{X,Y}(x,y)
=
\begin{cases}
a (x^2+y) & \text{ if } 0 < x < 1 \text{ and } 0 < y < 1\\
0 & \text{ otherwise} \enspace .
\end{cases}
\]
Find the following:
\be
\item~the normalizing constant $a$ which will ensure $\P(\Omega) = \int_{-\infty}^{\infty}\int_{-\infty}^{\infty} f_{X,Y}(x,y) dx dy = 1$
\item~$f_X(x) = \int_{-\infty}^{\infty} f_{X,Y}(x,y) dy$ called the marginal probability density function (MPDF) of $X$
\item~$f_Y(y) = \int_{-\infty}^{\infty} f_{X,Y}(x,y) dx$ called the marginal probability density function (MPDF) of $Y$
\item Check if $f_X(x)f_Y(y)=f_{X,Y}(x,y)$ for every $(x,y)$ and decide whether $X$ and $Y$ are independent random variables.  {Hint: $X$ and $Y$ are said to be independent if $f_X(x)f_Y(y)=f_{X,Y}(x,y)$ for every $(x,y)$.}
\item~$F_{X,Y}(x,y)$, the joint cumulative distribution function (JCDF) of $(X,Y)$ for any $(x,y) \in (0,1) \times (0,1)$ 
\item~the probability that $X > 0.5$ and $Y<0.6$, i.e., $\P(X>0.5,Y<0.6)$
\item~$E(X)$, the expectation of $X$ or the first moment of $X$
\item~$E(Y)$, the expectation of $Y$ or the first moment of $Y$
\item~$E(XY)$, the expectation of $XY$
\item~$\cv(X,Y)=E(XY)-E(X)E(Y)$, the covariance of $X$ and $Y$.
\ee

\Answer
~\\
\be
\item~To find $a$ we simply set $1=\int_{-\infty}^{\infty}\int_{-\infty}^{\infty} f_{X,Y}(x,y) dx dy$ and solve for $a$ as follows:
\begin{align*}
1
&=
\int_{-\infty}^{\infty}\int_{-\infty}^{\infty} f_{X,Y}(x,y) dx dy = \int_{0}^{1}\int_{0}^{1} a(x^2+y) dx dy\\
&=
a\int_{0}^{1}\int_{0}^{1} (x^2+y) dx dy = a\int_{0}^{1} \left[ \frac{1}{3}x^3+yx \right]_{x=0^1} dy\\
&=
a\int_{0}^{1} \left( \frac{1}{3}+y - 0 \right) dy = a \left[ \frac{y}{3}+\frac{1}{2}y^2 \right]_{y=0}^{1}\\
&=
a \left( 0-\left(\frac{1}{3}+\frac{1}{2}1^2\right) \right) = a \left( \frac{1}{3}+\frac{1}{2} \right)\\
&=
a \left( \frac{5}{6} \right)
\end{align*}
Therefore $a=6/5$ and the joint PDF is
\[
f_{X,Y}(x,y)
=
\begin{cases}
\frac{6}{5} (x^2+y) & \text{ if } 0 < x < 1 \text{ and } 0 < y < 1\\
0 & \text{ otherwise} \enspace .
\end{cases}
\]
\item~First compute the marginal PDF $f_X(x)$ for any $x \in (0,1)$ by integrating over $y$
\begin{align*}
f_X(x) 
&= \int_{-\infty}^{\infty} f_{X,Y}(x,y) dy
= \int_0^1 \frac{6}{5} (x^2+y) dy 
= \left[ \frac{6}{5} (yx^2+y^2/2) \right]_{y=0}^1\\
&= \frac{6}{5} \left((1 \times x^2 + 1^2/2) - 0 \right)
= \frac{6}{5} \left( x^2 + \frac{1}{2} \right)
\end{align*}
Finally, the marginal PDF of the RV $X$ in the first component of the \rv~$(X,Y)$ is
\[
f_{X}(x)
=
\begin{cases}
\frac{6}{5} \left( x^2 + \frac{1}{2} \right) & \text{ if } 0 < x < 1\\
0 & \text{ otherwise} \enspace .
\end{cases}
\]
\item~Similarly, the marginal PDF $f_Y(y)$ for any $y \in (0,1)$ by integrating over $x$ is 
\[
f_Y(y)=\int_{-\infty}^{\infty} f_{X,Y}(x,y) dx = \int_0^1 \frac{6}{5}(x^2+y) dx = \frac{6}{5} \left[ x^3/3+yx\right]_{x=0}^1 = \frac{6}{5}(y+1/3).
\]
Finally, the marginal PDF of the RV $Y$ in the second component of the \rv~$(X,Y)$ is
\[
f_{Y}(y)
=
\begin{cases}
\frac{6}{5} \left( y + \frac{1}{3} \right) & \text{ if } 0 < y < 1\\
0 & \text{ otherwise} \enspace .
\end{cases}
\]
\item The product of marginal PDFs of $X$ and $Y$ does not equal the joint PDF of $(X,Y)$ for values of $(x,y) \in (0,1)^2$
\[
f_X(x)f_Y(y) = \frac{6}{5} \frac{6}{5} \left( y + \frac{1}{3} \right) \left( x^2 + \frac{1}{2} \right) = \frac{6}{25} \left( 6 x^2y+2x^2+3y+1\right) \neq \frac{6}{5}(x^2+y) = f_{X,Y}(x,y)
\]
Therefore $X$ and $Y$ are not independent random variables (they are dependent!).
\item~The joint distribution function $F_{X,Y}(x,y)$ for any $(x,y) \in (0,1)^2$ is
\begin{align*}
F_{X,Y}(x,y) 
&= \int_{-\infty}^{y} \int_{-\infty}^{x} f_{X,Y}(u,v) du dv
= \int_{0}^{y} \int_{0}^{x} \frac{6}{5} (u^2+v) du dv
= \frac{6}{5} \int_{0}^{y} \left[ \frac{u^3}{3}+vu) \right]_{u=0}^{x} dv\\
&= \frac{6}{5} \int_{0}^{y} \left( \frac{x^3}{3}+vx - 0 \right) dv
= \frac{6}{5} \left[ \frac{x^3v}{3}+\frac{v^2x}{2} \right]_{v=0}^{y}
= \frac{6}{5} \left( \frac{x^3y}{3}+\frac{y^2x}{2} - 0 \right)\\
&= \frac{6}{5} \left( \frac{x^3y}{3}+\frac{y^2x}{2} - 0 \right)
\end{align*}
\item~
\begin{align*}
\P(X>0.5,Y<0.6) 
&= \int_{-\infty}^{0.6} \int_{0.5}^{\infty} f_{X,Y}(x,y) dx dy
= \int_{0}^{0.6} \int_{0.5}^{1} \frac{6}{5} (x^2+y) dx dy\\
&= \frac{6}{5} \int_{0}^{0.6} \left[ \frac{x^3}{3}+yx \right]_{x=0.5}^{1} dy
= \frac{6}{5} \int_{0}^{0.6} \left(\frac{7}{24} + \frac{y}{2} \right) dy\\
&= \frac{6}{5} \left[\frac{7}{24}y + \frac{y^2}{2} \right]_{y=0}^{0.6}
= \frac{6}{5} \left(\frac{7}{24}\times \frac{6}{10} + \frac{36}{400} \right)
=0.318
\end{align*}
\item~
\begin{align*}
E(X) 
&= \int_{-\infty}^{\infty} \int_{-\infty}^{\infty} x f_{X,Y}(x,y) dx dy\\
&=\frac{6}{5} \int_{0}^{1} \int_{0}^{1}  x \left( x^2 + y \right) dx dy
=\frac{6}{5} \int_{0}^{1} \int_{0}^{1}  \left( x^3 + xy \right) dx dy
= \frac{6}{5} \int_{0}^{1} \left[ \frac{x^4}{4} + \frac{1}{2}x^2y\right]_{x=0}^{1} dy\\
&=\frac{6}{5} \int_{0}^{1} \left( \frac{1}{4} + \frac{y}{2} - 0 - 0\right) dy
= \frac{6}{5} \left[ \frac{y}{4} + \frac{y^2}{4} \right]_{y=0}^{1} 
= \frac{6}{5} \left(\frac{1}{4} + \frac{1}{4} - 0 - 0 \right) = \frac{6}{5}\times \frac{1}{2}=\frac{3}{5}
\end{align*}
\item~
\begin{align*}
E(Y) 
&= \int_{-\infty}^{\infty} \int_{-\infty}^{\infty} y f_{X,Y}(x,y) dx dy\\
&= \int_{0}^{1} \int_{0}^{1} y \frac{6}{5} \left( x^2 + y \right) dx dy
= \frac{6}{5} \int_{0}^{1} \int_{0}^{1}  \left( x^2y + y^2 \right) dx dy
= \frac{6}{5} \int_{0}^{1} \left[ \frac{x^3y}{3} + y^2x \right]_{x=0}^{1} dy\\
&= \frac{6}{5} \int_{0}^{1} \left( \frac{y}{3} + y^2 -0-0\right) dy
= \frac{6}{5} \left[ \frac{y^2}{6} + \frac{y^3}{3} \right]_{y=0}^{1} 
= \frac{6}{5} \left( \frac{1}{6} + \frac{1}{3} + - 0 -0 \right) = \frac{6}{5} \times \frac{3}{6}=\frac{3}{5}
\end{align*}
\item~
\begin{align*}
E(XY) 
&= \int_{-\infty}^{\infty} \int_{-\infty}^{\infty} xy f_{X,Y}(x,y) dx dy\\
&=\frac{6}{5} \int_{0}^{1} \int_{0}^{1}  xy \left( x^2 + y \right) dx dy
=\frac{6}{5} \int_{0}^{1} \int_{0}^{1} x^3y+xy^2 dx dy
=\frac{6}{5} \int_{0}^{1} \left[ \frac{x^4}{4}y+\frac{x^2 y^2}{2} \right]_{x=0}^1 dy\\
&=\frac{6}{5} \int_{0}^{1} \left( \frac{y}{4} + \frac{y^2}{2} -0-0\right)dy
=\frac{6}{5} \left[ \frac{y^2}{8} + \frac{y^3}{6}\right]_{y=0}^1 
=\frac{6}{5} \left(\frac{1}{8} + \frac{1}{6}-0-0\right)\\
&=\frac{6}{5} \left(\frac{3}{24}+\frac{4}{24}\right)=\frac{6}{5}\times\frac{7}{24}=\frac{7}{20}
\end{align*}
\item~
\[
\cv(X,Y) = E(XY)-E(X)E(Y) = \frac{7}{20}-\left(\frac{3}{5} \times \frac{3}{5} \right) = \frac{7}{20}-\frac{9}{25} = \frac{35}{100}-\frac{36}{100} = -\frac{1}{100}
\]
\ee

\Exercise
Logs are milled to have a width of $\mu$.
The actual width of a randomly selected item is $X$.
If $X$ is a $\normal(\mu,\sigma^2)$ random variable then find the probability density function of the {\em squared-error} of the milling process,
\[ Y\,=\, (X-\mu)^2\,.\]
\Answer
Note that $Y= (X-\mu)^2$ is not on-to-one so it is better to use the direct method by differentiating the distribution function  of $Y$, $F_Y(y)$,  to obtain $f_Y(y)$.

If $y\geqslant 0$,
\begin{align*}
 F_Y(y)\;=&\; \P(Y \leqslant y) \\[3pt]
 \;=&\;  \P( (X-\mu)^2 \leqslant y) \\[3pt]
 \;=&\; \P(-\sqrt y \leqslant X-\mu \leqslant \sqrt y)\\[3pt]
 \;=&\; \P(\mu -\sqrt y \leqslant X  \leqslant \mu + \sqrt y)\\[3pt]
\;=&\; F_X( \mu + \sqrt y) \,-\, F_X( \mu + \sqrt y)
\end{align*}

Differentiating this expression gives
\begin{align*}
f_Y(y) \;=&\; \frac{d}{dy} \left(F_X( \mu + \sqrt y) \,-\, F_X( \mu - \sqrt y)  \right) \\[3pt]
\;=&\;\frac{1}{2} y^{-\frac{1}{2}} \, f_X (\mu + \sqrt y ) \,- \, \left( -\frac{1}{2} y^{-\frac{1}{2}} \right)\, f_X (\mu - \sqrt y ) \\[3pt]
\;=&\;\frac{1}{2 \sqrt y } \left(  f_X (\mu + \sqrt y )  + f_X (\mu - \sqrt y )  \right)
\end{align*}

Note: If $y<0$ then $f_Y(y) = 0$  since  $F_Y(y)=0$ in this case.

\Exercise
Let $(X,Y)$ be a discrete random vector (\rv) with support:
\[
\mathcal{S}_{X,Y} = \{(0,0),(0,1),(1,0),(1,1)\} \enspace .
\]
Let its joint probability mass function (JPMF) be:
\[
f_{X,Y}(x,y) = 
\begin{cases}
\frac{1}{4} & \text{ if } (x,y)=(0,0)\\
\frac{1}{4} & \text{ if } (x,y)=(0,1)\\
\frac{1}{4} & \text{ if } (x,y)=(1,0)\\
\frac{1}{4} & \text{ if } (x,y)=(1,1)\\
0 & \text{ otherwise} \enspace .
\end{cases}
\]
Are $X$ and $Y$ independent? 
\Answer
~\\
First derive the marginal PMF of $X$ and $Y$ and then check if the JPMF is the product of the marginal PMFs.
\[
f_X(0) = \sum_{y \in \mathcal{S}_{X,Y}} f_{X,Y}(0,y) = f_{X,Y}(0,0) + f_{X,Y}(0,1) = \frac{1}{4}+\frac{1}{4}=\frac{1}{2}
\]
and
\[
f_X(1) = \sum_{y \in \mathcal{S}_{X,Y}} f_{X,Y}(1,y) = f_{X,Y}(1,0) + f_{X,Y}(1,1) = \frac{1}{4}+\frac{1}{4}=\frac{1}{2}
\]
Thus,
\[
f_{X}(x) = 
\begin{cases}
\frac{1}{2} & \text{ if } x = 0\\
\frac{1}{2} & \text{ if } x = 1\\
0 & \text{ otherwise} \enspace .
\end{cases}
\]
Similarly,
\[
f_{Y}(y) = 
\begin{cases}
\sum_{x \in \mathcal{S}_{X,Y}} f_{X,Y}(x,0)=f_{X,Y}(0,0) + f_{X,Y}(1,0) = \frac{1}{4}+\frac{1}{4}=\frac{1}{2} & \text{ if } y = 0\\
\sum_{x \in \mathcal{S}_{X,Y}} f_{X,Y}(x,1)=f_{X,Y}(0,1) + f_{X,Y}(1,1) = \frac{1}{4}+\frac{1}{4}=\frac{1}{2} & \text{ if } y = 1\\
0 & \text{ otherwise} \enspace .
\end{cases}
\]
Finally, the product of $f_X(x)$ and $f_Y(y)$ is
\[
f_{X}(x) \times f_{Y}(y) = 
\begin{cases}
\frac{1}{2} \times \frac{1}{2}=\frac{1}{4} & \text{ if } (x,y)=(0,0)\\
\frac{1}{2} \times \frac{1}{2}=\frac{1}{4} & \text{ if } (x,y)=(0,1)\\
\frac{1}{2} \times \frac{1}{2}=\frac{1}{4} & \text{ if } (x,y)=(1,0)\\
\frac{1}{2} \times \frac{1}{2}=\frac{1}{4} & \text{ if } (x,y)=(1,1)\\
0 & \text{ otherwise} \enspace .
\end{cases}
\]
which in turn is equal to the JPMF $f_{X,Y}(x,y)$ in the question.  Therefore we have shown that the component RVs $X$ and $Y$ in the \rv~$(X,Y)$ are indeed indepedent.

\Exercise
A semiconductor product consists of three layers that are fabricated independently.  
If the variances in thickness of the first, second and third third layers are $25$, $40$ and $30$ nanometers squared, what is the variance of the thickness of the final product? 
\Answer
~\\
Let $X_1$, $X_2$, $X_3$ be independent RVs that denote the thickness of the first, second and third layer, respectively.  Let $X$ denote the thickness of the final product.  Then
\[
X= X_1+X_2+X_3
\]
By the property that $V\left(\sum_{i=1}^n a_i X_i\right) = \sum_{i=1}^n a_i^2 V(X_i)$, Variance of $X$ is
\[
V(X) = 1^2 V(X_1) + 1^2 V(X_2) + 1^2 V(X_3) = 25+40+30 = 95 nm^2 \enspace .
\] 
This shows how the variance in each layer is propagated to the variance of the final product.

\Exercise
Find the covariance for the discrete \rv~$(X,Y)$ with joint probability mass function
\[
f_{X,Y}(x,y) = 
\begin{cases}
0.2 & \text{ if } (x,y)=(0,0)\\
0.1 & \text{ if } (x,y)=(1,1)\\
0.1 & \text{ if } (x,y)=(1,2)\\
0.1 & \text{ if } (x,y)=(2,1)\\
0.1 & \text{ if } (x,y)=(2,2)\\
0.4 & \text{ if } (x,y)=(3,3)\\
0 & \text{ otherwise } \enspace .
\end{cases}
\]
[Hint: Recall that $\cv(X,Y)=E(XY)-E(X)E(Y)$]
\Answer
~\\
Find $E(XY)$, $E(X)$ and $E(Y)$ to get $\cv(X,Y)=E(XY)-E(X)E(Y)$ as follows:
\begin{multline*}
E(XY) = \sum_{(x,y) \in \mathcal{S}_{X,Y}} x \times y \times f_{X,Y}(x,y) \\
= 0 \times 0 \times 0.2 + 1 \times 1 \times 0.1+ 1 \times 2 \times 0.1 + 2 \times 1 \times 0.1 + 2 \times 2 \times 0.1 + 3 \times 3 \times 0.4 = 4.5
\end{multline*}
\begin{multline*}
E((X,Y)) = \sum_{(x,y) \in \mathcal{S}_{X,Y}} (x , y) \times f_{X,Y}(x,y) \\
= (0 , 0) \times 0.2 + (1 , 1) \times 0.1 + (1 , 2) \times 0.1 + (2 , 1) \times 0.1 + (2 , 2) \times 0.1 + (3 , 3) \times 0.4\\= (0,0)+(0.1,0.1)+(0.1,0.2)+(0.2,0.1)+(0.2,0.2)+(1.2,1.2) \\
= (1.8,1.8) 
\end{multline*}
Since addition is component-wise $E((X,Y))=(E(X),E(Y))$ and therefore $E(X)=E(Y)=1.8$.

Alternatively, you can first find the marginal PMFs $f_X$ and $f_Y$ for $X$ and $Y$ and then take the expectations $E(X)=\sum_x x \times f_X(x)$ and $E(Y)=\sum_y y \times f_Y(y)$.

Finally, 
\[
\cv(X,Y) = E(XY)-E(X)E(Y) = 4.5 - 1.8^2 = 1.26 \enspace .
\]

\Exercise
Consider two random variables (RVs) $X$ and $Y$ having marginal distribution functions
\[
F_X(x) =
\begin{cases}
0 & \text{ if } x < 1\\
\frac{1}{2} & \text{ if } 1 \leq x < 2\\
1 & \text{ if } x \geq 2\\
\end{cases}
\] 
\[
F_Y(y) =
\begin{cases}
0 & \text{ if } y < 0\\
1-\frac{1}{2}e^{-y}-\frac{1}{2}e^{-2y} & \text{ if } y \geq 0\\
\end{cases}
\]
If $X$ and $Y$ are independent, what is their joint distribution function $F_{X,Y}(x,y)$? [Hint: you need to express $F_{X,Y}(x,y)$ for any $(x,y) \in \Rz^2$.]
\Answer
~\\
Since $X$ and $Y$ are independent, $F_{X,Y}(x,y) = F_X(x) F_Y(y)$ for all $(x,y) \in \Rz^2$, and we get:
\[
F_{X,Y}(x,y) = 
\begin{cases}
0 & \text{ if } x < 1 \text{ or } y < 0\\
\frac{1}{2}\left(1-\frac{1}{2}e^{-y}-\frac{1}{2}e^{-2y}\right) & \text{ if } 1 \leq x < 2 \text{ and } y \geq 0\\
1-\frac{1}{2}e^{-y}-\frac{1}{2}e^{-2y} & \text{ if } x \geq 2 \text{ and } y \geq 0
\end{cases}
\]
{\small
You can arrive at the answer by partitioning $x$-axis into $(-\infty,1)$, $[1,2)$ and $[2,\infty)$ where $F_X(x)$ takes distinct values.  Similarly, partition the $y$-axis into  $(-\infty,0)$ and $[0,\infty)$ where $F_Y(y)$ takes distinct values.  
Now $(x,y)$ can take values in one of these $3 \times 2=6$ partitions of the $x \times y$ plane as follows (make a picture!):
\[
(-\infty,1) \times (-\infty,0), \,
[1,2) \times (-\infty,0), \,
[2,\infty) \times (-\infty,0), \,
(-\infty,1) \times [0,\infty), \,
[1,2) \times [0,\infty), \,
[2,\infty) \times [0,\infty) \enspace .
\]
Now work out what $F_{X,Y}(x,y) = F_X(x) F_Y(y)$ is for $(x,y)$ in each of the above six partitions of the plane and you will get the the expression for $F_{X,Y}(x,y)$ given above.
}

\Exercise
Let $(X,Y)$ be a continuous \rv~with joint probability density function (JPDF):
\[
f_{X,Y}(x,y) = 
\begin{cases}
e^{-x} & \text{ if } x \in [0,\infty) \text{ and }  y \in [2,3]\\
0 & \text{otherwise} \enspace .
\end{cases}
\]
Are $X$ and $Y$ independent?
\Answer
~\\
First obtain marginal PDF of $Y$.  If $y \in [2,3]$ then
\[
f_Y(y) = \int_{-\infty}^{\infty} f_{X,Y}(x,y) dx = \int_0^{\infty} e^{-x} dx = \left[ -e^{-x} \right]_0^{\infty} = 0 - (-1)=1 \enspace .
\]
Therefore,
\[
f_Y(y) =
\begin{cases}
1 & \text{ if } y \in [2,3]\\
0 & \text{otherwise} \enspace .
\end{cases}
\]
Now, obtain the marginal PDF of $X$.  If $x \in [0,\infty)$ then
\[
f_X(x) = \int_{-\infty}^{\infty} f_{X,Y}(x,y) dy = \int_2^3 e^{-x} dy  = e^{-x} \int_2^3 1 dy = e^{-x} \left[ y \right]_2^3 = e^{-x} (3-2) = e^{-x} \enspace .
\]
Therefore,
\[
f_X(x) =
\begin{cases}
e^{-x} & \text{ if } x \in [0,\infty)\\
0 & \text{otherwise} \enspace .
\end{cases}
\]
Finally, verifying that $f_{X,Y}(x,y) = f_X(x) f_Y(y)$ for any $(x,y) \in \Rz^2$ is done case by case.  
Draw a picture on the plane to work out the cases from the distinct expressions taken by $f_{X,Y}(x,y)$.  
There are only two cases to consider (when $f_{X,Y}(x,y)$ takes zero values and when $f_{X,Y}(x,y)$ takes non-zero values):
\be
\item~If $x \notin [0,\infty)$ or $y \notin [2,3]$ then $f_X(x)f_Y(y)=0=f_{X,Y}(x,y)$
\item~If $x \in [0,\infty)$ and $y \in [2,3]$ then $f_X(x)f_Y(y)=e^{-x} \times 1=e^{-x}=f_{X,Y}(x,y)$.
\ee
Thus, $X$ and $Y$ are independent.

\Exercise
In an electronic assembly, let the RVs $X_1,X_2,X_3,X_4$ denote the lifetimes of four components in hours.  
Suppose that the JPDF of these variables is
\begin{align*}
~& f_{X_1,X_2,X_3,X_4}(x_1,x_2,x_3,x_4) \\
&\qquad= 
\begin{cases}
9 \times 10^{-12} e^{-0.001 x_1 - 0.002 x_2 - 0.0015 x_3 - 0.003 x_4} & \text{ if } x_1 \geq 0, x_2 \geq 0, x_3 \geq 0, x_4 \geq 0\\
0 & \text{ otherwise} \enspace.
\end{cases}
\end{align*}
What is the probability that the device operates for more than 1000 hours without any failures? [Hint: The requested probability is $\P(X_1>1000,X_2>1000,X_3>1000,X_4>1000)$ since each one of the four components of the device must not fail before 1000 hours.]
\Answer
~\\
\begin{multline*}
\P(X_1>1000,X_2>1000,X_3>1000,X_4>1000) \\
= \int_{1000}^{\infty} \int_{1000}^{\infty} \int_{1000}^{\infty} \int_{1000}^{\infty} 9 \times 10^{-12} e^{-0.001 x_1 - 0.002 x_2 - 0.0015 x_3 - 0.003 x_4} dx_1 dx_2 dx_3 dx_4\\
= 9 \times 10^{-12} \int_{1000}^{\infty} \int_{1000}^{\infty} \int_{1000}^{\infty} \int_{1000}^{\infty}  e^{-0.001 x_1} e^{- 0.002 x_2} e^{- 0.0015 x_3} e^{- 0.003 x_4} dx_1 dx_2 dx_3 dx_4\\
=  9 \times 10^{-12} \int_{1000}^{\infty} e^{-0.001 x_1} \int_{1000}^{\infty} e^{- 0.002 x_2} \int_{1000}^{\infty} e^{- 0.0015 x_3} \int_{1000}^{\infty} e^{- 0.003 x_4} dx_4 dx_3 dx_2 dx_1\\
\end{multline*}
Since
\[
\int_{1000}^{\infty} e^{- a x_i} dx_i = \left[ \frac{e^{-a x_i}}{-a} \right]_{1000}^{\infty} = 0 + \frac{e^{-1000\times a}}{a} \enspace ,
\]
the above quadruply iterated integral becomes
\begin{align*}
&~ 9 \times 10^{-12} \times \frac{e^{-1000 \times 0.001}}{0.001} \times \frac{e^{-1000 \times 0.002}}{0.002} 
\times \frac{e^{-1000 \times 0.0015}}{0.0015} \times \frac{e^{-1000 \times 0.003}}{0.003}\\
&= 9 \times 10^{-12} \times \frac{1000}{1} \times \frac{1000}{2} \times \frac{1000}{1.5} \times \frac{1000}{3} \times
e^{-1} \times e^{-2} \times e^{-1.5} \times e^{-3} \\
&= 9 \times 10^{-12} \times \frac{1}{9} \times 10^{12} \times e^{-7.5} = e^{-7.5} \approxeq 0.00055 \enspace .
\end{align*}

\Exercise
Suppose the RVs $Y_1$, $Y_2$ and $Y_3$ represent the thickness in micrometers of a substrate, an active layer, and a coating layer of a chemical product.  
Assume $Y_1$, $Y_2$ and $Y_3$ are $\normal(10000,250^2)$, $\normal(1000,20^2)$ and $\normal(80,4^2)$ RVs, respectively.  
Further suppose that they are independent.  
The required specifications for the thickness of the substrate, active layer and coating layer are $[9500,10500]$, $[950,1050]$ and $[75,85]$, respectively.  
What proportion of chemical products meets all thickness specifications? [Hint: this is just $\P(9500<Y_1<10500,950<Y_2<1050,75<Y_3<85)$]  Which one of the three thicknesses has the least probability of meeting specifications?   
\Answer
~\\
Due to independence of $Y_1$, $Y_2$ and $Y_3$
\begin{multline*}
\P(9500<Y_1<10500,950<Y_2<1050,75<Y_3<85)\\ = \P(9500<Y_1<10500) \P(950<Y_2<1050) \P(75<Y_3<85)
\end{multline*}
After standardizing each Normal RV (subtracting its mean and dividing by its standard deviation) we get
\begin{align*}
&~ \P(9500<Y_1<10500) \P(950<Y_2<1050) \P(75<Y_3<85)\\
&=  P\left(\frac{9500-10000}{250}< Z < \frac{10500-10000}{250} \right) P\left( \frac{950-1000}{20}<Z<\frac{1050-1000}{20}\right)\\
&\qquad \qquad \qquad \qquad \qquad \qquad P\left(\frac{75-80}{4}<Z<\frac{85-80}{4}\right)\\
&= \P(-2.0<Z<2.0) \P(-2.5<Z<2.5) \P(-1.25<Z<1.25) \\
&= (\Phi(2.0)-(1-\Phi(2.0)) \times (\Phi(2.5)-(1-\Phi(2.5)) \times (\Phi(1.25)-(1-\Phi(1.25))\\
&= (2\Phi(2.0)-1) \times (2\Phi(2.5)-1) \times (2\Phi(1.25)-1)\\
&= ((2 \times 0.9772) -1) \times ((2 \times 0.9938) - 1) \times ((2 \times 0.8944)-1) \qquad \text{ using Table for $\Phi(z)$} \\
&= 0.9544 \times 0.9876 \times 0.7888 = 0.7435
\end{align*}
%To follow the fourth-last equality above see Example 8.14(d) from EMTH119. % TODO? 
The values for the distribution function $\Phi(z)$ of the $\normal(0,1)$ RV $Z$ are in the table on page 67. 

The thickness of the coating layer represented by $Y_3$ has the least probability ($0.7888$) of meeting specifications.  Consequently, a priority should be to reduce variability in this part of the process.

\Exercise
Soft drink cans are filled by an automated filling machine.  
Assume the fill volumes of the cans are independent $\normal(12.1,0.01)$ RVs.  
What is the probability that the average volume of ten cans selected from this process is less than $12.01$ fluid ounces?
\Answer
~\\
Let $X_1,X_2,\ldots,X_{10}$ denote the fill volumes of $10$ cans.  The average fill volume is the sample mean
\[
\ol{X}_{10} = \frac{1}{10} \sum_{i=1}^{10} X_i
\]
By property of Expectations and Variances for linear combinations
\[
E(\ol{X}_{10}) = E \left(\frac{1}{10} \sum_{i=1}^{10} X_i \right) = \frac{1}{10}\sum_{i=1}^nE(X_i) = \frac{1}{10}\sum_{i=1}^nE(X_1) = \frac{1}{10} \times 10 \times E(X_1) = E(X_1) = 12.1
\]
Or by directly using the ``formula'' $E(\ol{X}_{10}) = E(X_1)=12.1$ for these $10$ identically distributed RVs.  
Similarly, 
\[
V\left(\ol{X}_{10}\right) = V \left(\frac{1}{10} \sum_{i=1}^{10} X_i \right) = 10 \times \frac{1}{10^2} V(X_1) = \frac{1}{10} \times 0.01 = 0.001 
\]
Or by directly using the ``formula'' $V(\ol{X}_{10}) = V(X_1)/10$ for these $10$ independently and identically distributed RVs.
 
By the special property of Normal RVs -- a linear combination of independent normal RVs is also normal -- we know that 
$\ol{X}_{10}$ is a $\normal(12.1,0.001)$ RV.
Consequetly, the probability of interest is
\begin{align*}
\P(\ol{X}_{10} < 12.01) 
&= P \left( \frac{\ol{X}_{10}-E(\ol{X}_{10})}{\sqrt{0.001}} < \frac{12.01 - E(\ol{X}_{10})}{\sqrt{0.001}} \right) = P\left(Z < \frac{12.01-12.1}{0.0316}\right)\\ 
&\approxeq \P(Z<-2.85) = 1-\P(Z<2.85) = 1-\Phi(2.85) = 1-0.9978=0.0022
\end{align*}

\Exercise
Let $X_1,X_2,X_3,X_4$ be RVs that denote the number of bits received in a digital channel that are classified as {\em excellent}, {\em good}, {\em fair} and {\em poor}, respectively.  
In a transmission of $10$ bits, what is the probability that $6$ of the bits received are {\em excellent}, $2$ are {\em good}, $2$ are {\em fair} and none are {\em poor} under the assumption that the classification of bits are independent events and that the probabilities of each bit being {\em excellent}, {\em good}, {\em fair} and {\em poor} are $0.6$, $0.3$, $0.08$ and $0.02$, respectively. 
[Hint: Think of $\multinomial(n=10,\theta_1=0.6,\theta_2=0.3,\theta_3=0.08,\theta_4=0.02)$ as a model for bit classification in this digital channel.]
\Answer
~\\
Using the $\multinomial(n=10,\theta_1=0.6,\theta_2=0.3,\theta_3=0.08,\theta_4=0.02)$ \rv~as our model
\begin{align*}
&~ P\left( (X_1,X_2,X_3,X_4)=(6,2,2,0); n=10,\theta_1=0.6,\theta_2=0.3,\theta_3=0.08,\theta_4=0.02 \right)\\ 
&= \frac{10!}{6! \times 2! \times 2! \times 0!} \times 0.6^6 \times 0.3^2 \times 0.08^2 \times 0.02^0\\
&= \frac{10 \times 9 \times 8 \times 7 \times 6 \times 5 \times 4 \times 3 \times 2 \times 1}{(6 \times 5 \times 4 \times 3 \times 2 \times 1) \times (2 \times 1) \times (2 \times 1) \times 1} \times  0.6^6 \times 0.3^2 \times 0.08^2 \times 1 \approxeq 0.03386\\
\end{align*}


\end{ExerciseList}

