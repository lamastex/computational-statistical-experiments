\RequirePackage[l2tabu,orthodox]{nag}
\documentclass[11pt,paper=a4,abstract=on,parskip=half,DIV=calc,compact]{scrartcl}
\usepackage{silence} \WarningFilter{fixltx2e}{}

\usepackage[utf8]{inputenc}
\usepackage[T1]{fontenc}
% Latin Modern font, extended version of the default Computer Modern font 
\usepackage{lmodern}

% \usepackage{microtype}
% \usepackage[activate={true,nocompatibility},final,tracking=true,kerning=true,spacing=true,factor=1100,stretch=10,shrink=10]{microtype}

\usepackage{scrlayer-scrpage}

\usepackage[shortlabels]{enumitem}

\usepackage{booktabs}

\usepackage{array}

\usepackage[usenames,dvipsnames,svgnames,table]{xcolor}

\usepackage{subcaption}

\usepackage[swedish]{babel}

\usepackage[all,warning]{onlyamsmath}
\usepackage{mathtools}
\usepackage{amssymb}
\usepackage{amsthm}

\usepackage[separate-uncertainty=true,multi-part-units=single]{siunitx}

\usepackage{graphicx}

\usepackage{tikz}
\usetikzlibrary{babel}

\usepackage{csquotes}

\usepackage{float}

\usepackage{todonotes}

\usepackage{listings}
\lstset{literate=
  {á}{{\'a}}1 {é}{{\'e}}1 {í}{{\'i}}1 {ó}{{\'o}}1 {ú}{{\'u}}1
  {Á}{{\'A}}1 {É}{{\'E}}1 {Í}{{\'I}}1 {Ó}{{\'O}}1 {Ú}{{\'U}}1
  {à}{{\`a}}1 {è}{{\`e}}1 {ì}{{\`i}}1 {ò}{{\`o}}1 {ù}{{\`u}}1
  {À}{{\`A}}1 {È}{{\'E}}1 {Ì}{{\`I}}1 {Ò}{{\`O}}1 {Ù}{{\`U}}1
  {ä}{{\"a}}1 {ë}{{\"e}}1 {ï}{{\"i}}1 {ö}{{\"o}}1 {ü}{{\"u}}1
  {Ä}{{\"A}}1 {Ë}{{\"E}}1 {Ï}{{\"I}}1 {Ö}{{\"O}}1 {Ü}{{\"U}}1
  {â}{{\^a}}1 {ê}{{\^e}}1 {î}{{\^i}}1 {ô}{{\^o}}1 {û}{{\^u}}1
  {Â}{{\^A}}1 {Ê}{{\^E}}1 {Î}{{\^I}}1 {Ô}{{\^O}}1 {Û}{{\^U}}1
  {œ}{{\oe}}1 {Œ}{{\OE}}1 {æ}{{\ae}}1 {Æ}{{\AE}}1 {ß}{{\ss}}1
  {ű}{{\H{u}}}1 {Ű}{{\H{U}}}1 {ő}{{\H{o}}}1 {Ő}{{\H{O}}}1
  {ç}{{\c c}}1 {Ç}{{\c C}}1 {ø}{{\o}}1 {å}{{\r a}}1 {Å}{{\r A}}1
  {€}{{\euro}}1 {£}{{\pounds}}1 {«}{{\guillemotleft}}1
  {»}{{\guillemotright}}1 {ñ}{{\~n}}1 {Ñ}{{\~N}}1 {¿}{{?`}}1
}

% \usepackage{ragged2e}

\let\originalleft\left
\let\originalright\right
\renewcommand{\left}{\mathopen{}\mathclose\bgroup\originalleft}
\renewcommand{\right}{\aftergroup\egroup\originalright}

% \usepackage[backend=biber]{biblatex}
% \addbibresource{refs.bib}

% \usepackage{tcolorbox}
% \usepackage{etoolbox}

% \makeatletter{}
%   \patchcmd{\@sect}{}
% \makeatother{}

\usepackage[colorlinks=false, pdfborder={0 0 0}]{hyperref}

\usepackage[noabbrev]{cleveref}
\usepackage[lastexercise,answerdelayed]{exercise}

\recalctypearea

\pagestyle{plain.scrheadings}

% \pagestyle{scrheadings}
% \renewcommand{\titlepagestyle}{scrheadings}

% \setheadtopline{1pt}
% \setheadsepline{0.5pt}
% \setfootsepline{0.5pt}
% \setfootbotline{0.5pt}

% \ihead{One can}
% \chead{have}
% \ohead{stuff}
% \ifoot{all over}
% \cfoot{section: \thesection, page: \thepage}
% \ofoot{the place}

\newtheorem{theorem}{Theorem}
\newtheorem{corollary}{Corollary}
\newtheorem*{remark}{Remark}
\newtheorem{definition}{Definition}

\DeclareMathOperator{\supp}{supp}
\DeclareMathOperator{\erf}{erf}

\setcounter{secnumdepth}{0}

\newcommand{\points}[1]{\marginpar{(#1)}}


% to embed tutorial problems with the book at end of each chapter
%\newtoggle{PlaceTutsHere}
%\toggletrue{PlaceTutsHere}
%\togglefalse{PlaceTutsHere}

% to place tutorial solutions in the end of the book
%\newtoggle{PlaceTutsSolsHere}
%\toggletrue{PlaceTutsSolsHere}
%\togglefalse{PlaceTutsSolsHere}

\begin{document}

\thispagestyle{scrheadings}
\ihead{\textsc{Uppsala Universitet} \\ Matematiska Institutionen \\ Raazesh Sainudiin}
\chead{}
%\ohead{\textsc{Prov-Tentamen i Inferensteori I} \\  1MS035 \\ 2018-08-28}
\ohead{\textsc{Model Exam in Probability Theory I} \\  1MS034 \\ 20**-**-**} % Exam 0
\ifoot{}
\cfoot{}
\ofoot{}
\setheadsepline{1pt}

Time Allowed: 5 hours. Total sum: 40p. Grades 3, 4 and 5 require 18p, 25p and 32p, respectively.
%Skrivtid: 14:00--19:00. För betygen 3, 4 resp.\ 5 krävs 18, 25 resp.\ 32 poäng. Minst
%8 poäng måste vara från del A. Vis poängsättningen kommer en helhetsbedömning
%att göras. Observera:
Note:
\begin{itemize}[noitemsep,topsep=0pt,parsep=0pt,partopsep=0pt]
%\item \framebox{Motivera lösningar väl, dock ej mer än nödvändigt.}
	\item \framebox{Motivate solutions well but no more than necessary.}
  \vspace{1mm}
%\item Skriv endast på ena sidan av pappret.
\item Just write on one side of the paper.
%\item Påbörja ny uppgift på ny sida, skriv uppgiftens nummer på \emph{höger} sida.
\item Start new question on new page, and label the question number clearly next to your solution.
%\item Skriv ej med rödpenna.
\item Do not write with red-ink pen.
%\item Lägg lösningarna i \emph{stigande} nummerordning.
\item Write the solutions in increasing order of question numbers.
\item Hints and other information provided in a problem may also be useful for subsequent problems.
\item Submit this exam paper with your solutions, i.e., you may NOT take this exam paper with you.  
\end{itemize}
%Tillåtna hjälpmedel: Alla textböcker och anteckningar, miniräknar, %Räknedosa, eget A4-blad (fram- och baksida) med egenhändigt handskrivna antecknignar, 
%samt lexikon för översättning mellan svenska och andra språk. 
Permitted aids: Any books and notes.
\vspace{-0.2\baselineskip} \newline{}\rule{\textwidth}{1pt}

%\vspace{-1.5\baselineskip}

%\section{Del A. (totalsumma 12 p.\ minst 8 p.\ för godkänt)}

\vspace{\baselineskip}


\begin{ExerciseList}

\Exercise {\em 5p} --
Suppose you observe the following five data points from some product experiment:
\[
0,2,1,0,3 .
\]
\begin{enumerate}[(a)]
\item --- {\em 1p} -- Report the sample mean. 
\item --- {\em 1p} -- Report the sample variance. 
\item --- {\em 1p} -- Report the five order statistics from minimum to maximum.
\item --- {\em 1p} -- Sketch the plot of the Empirical Mass Function showing discontinuities in the function and clearly labeling the axes. 
\item --- {\em 1p} -- Sketch the plot of the Empirical Distribution Function showing discontinuities in the function and clearly labeling the axes. 
\end{enumerate}
\Answer
~\\
This was done in the last day. Note that you will get $0$ points if the points of discontinuity are not perfectly plotted with open and closed circles and solid and dashed lines with axes labelled and marked properly.
\begin{enumerate}[(a)]
\item sample mean = 
\vspace{2cm}
\item sample variance = 
\vspace{4cm}
\item order statistics is:
\vspace{2cm}
\item{Empirical Mass Function is given by:
\vspace{2cm}
\item{Empirical Distribution Function is given by:
\vspace{5cm}
}
\end{enumerate}

\Exercise {\em 5p} --
\begin{enumerate}
\item --- {\em 3p} -- 
{
In a survey, the two attributes of  houses, whether they have a chimney and/or have a tile roof,
were cross tabulated. The following table gives the resulting probabilities.
Are these two attributes independent? You must justify your answer.
$$\begin{array}{c|cc}
& \multicolumn{2}{c}{\mbox{Tile Roof}} \\
\mbox{Chimney} & \mbox{No} & \mbox{Yes}  \\\hline
\mbox{No}  & 0.3 & 0.1 \\
\mbox{Yes} & 0.4 & 0.2 \\
\end{array}$$
}
\item --- {\em 2p} --  
{
A certain river contains on average 1 trout per 100 m of length along the bank.
Assuming the trout occur at random and their location is independent of
other trout, what is the probability:
\begin{enumerate}
% \item There is one trout in a 100 m stretch of river.
\item There are two or more trout in a 100 m stretch of river.
\item There are no trout in a 500 m stretch of river.
\end{enumerate}
}
\end{enumerate}

\Answer
~\\
\begin{enumerate}%[(a)]
\item 
{For independence $P(A\cap B) = P(A)\,P(B)$ or equivalently $P(A | B) = P(A)$.
\begin{align*}
P(\,\text{Chimney}\,) &= 0.4+0.2 = 0.6\\
P(\,\text{Tile roof}\,) &= 0.1+0.2 = 0.3\\
P(\, \text{Chimney}\cap\text{Tile roof} ) &= 0.2 \neq 0.6 \times 0.3 = 0.18
\end{align*}
So the attributes are not independent.
}
\item
{
\begin{align*}
(a)\qquad \qquad P(\, X \geq 2\,) &= 1-P(0)-P(1) \\
& = 1 -e^{-1}-e^{-1} = 1 -2e^{-1} \\
& \approx 1 -2\times 0.368 =1 -0.736=0.264
\end{align*}
\vspace*{5mm}
\begin{align*}
(b)  \qquad \qquad \lambda =5, \qquad
P(\, X=0\,) &= e^{-5} \frac{5^0}{0!} = e^{-5}\\
& =0.007 \ \text{from the tables.   OR}\ (0.368)^5
\end{align*}
}
\end{enumerate}


\Exercise
 {\em 5 p} -- Assume 
{
In testing
 seeds for germination, batches of 20 seeds are tested at a time.
If the true probability of germination for a seed is 0.9, and the germination
of the seeds is independent, then:
\begin{enumerate}
\item[(i)] What is the expected number of seeds that will germinate?
\item[(ii)] What is the probability that exactly 19 seeds germinate.
\item[(iii)] Suppose there is a pollutant in the field and the probability of germination is $1/100$. Assuming germination is independent, find an expression for the probability that none of the $1000$ planted seeds in a given unit area of a field will germinate using the ``Poisson approximation to the Binomial''. 
\item[(iv)] Justify the probability model used in the above computation in (iii). Give an example of agricultural conditions that can violate this model? 
\item[(v)] Suppose there is another pollutant that reduces the probability of germination to $5/100$ independently of the presence of the first pollutant. Using Poisson approximations for the second pollutant (as well as the first pollutant), what is the probability that none of the $1000$ planted seeds in a unit area of a field will germinate in the presence of both pollutants?
\end{enumerate}
}
\Answer
{
(i) \qquad $\text{Expected number}\ = np = 20\times 0.9 = 18.$

(ii) \qquad $P( \, X = 19 \,) = \binom{20}{19}   (0.9)^{19}(0.1)^1 = 2 \, (0.9)^{19} $

(iii) [only partial answer and hints given]: ... [see Lacunae problem on steel specimens to figure out how to get the expression here...]

(iv) Heterogeneity in other field conditions like soil fertility, moisture, competing species, could violate the Poisson assumption.

(v) [only partial answer and hints given]: Get Poisson approxmation under second pollutant and recall Exercise 3.54 that the sum of independent Poissons is Poisson with paramater given by the sum of the individual Poisson parameters.
}

\Exercise  {\em 5p} --
\begin{enumerate}
\item
{Consider the continuous random variable, $X$, with the
probability density function
$$
f(x) = \left\{ \begin{array}{cl} \displaystyle Kx(1-x), & \mbox{if}~ 0 \leq x \leq 1   \\[2ex]
0, & \mbox{otherwise}
\end{array} \right..
$$
}
\begin{enumerate}
\item
{Find the value of $K$.}
\item
{Find the distribution function, $F(x)$.}
\item
{Find the probability, $P(\, 1/4<X<1/2 \,)$.}
\end{enumerate}
\item
{Consider the discrete random variable, $X$, with the
probability mass function given in the table below:\\
\[
\begin{array}{c|cccc}
x      & 0   &  1  &  2  & 4   \\ \hline
P(X=x) & 0.4 & 0.3 & 0.2 & 0.1 \\
\end{array}
\]
}
\begin{enumerate}
\item
{Find the mean value of $X$, $E(X)$.}
\item 
{If $Y = {1+X^2}$, find the expected value of $Y$, $E(Y)$.}
\end{enumerate}

\end{enumerate}

\Answer
\begin{enumerate}
\item
\begin{enumerate}
\item
{
To be a probability distribution
\begin{align*}
1 = \int_0^1 f(x)\, dx  &= K \int_0^1 x(1-x)\, dx\\
                        &= K \int_0^1 x -x^2 \, dx\\
                        & = K \left[ \frac{x^2}{2} - \frac{x^3}{3} \right]^1_0\\
                        & = K \frac{1}{6} .
\end{align*}
So that $1 =K/6 $, implying $K=6$.
}
\item
{
For $0 < x < 1$,
$$
F(x) = 6\left[ \frac{x^2}{2} - \frac{x^3}{3} \right] = 3x^2 -2x^3.
$$
Therefore,
$$F(x) = \begin{cases} 0, & x \leq 0\\
          3x^2 -2x^3, & 0 < x < 1\\
          1, & x \geq 1.
          \end{cases}
$$ }
\item
{
\begin{align*} P(\, \frac{1}{4} < x < \frac{1}{2} \, )
& = F\left( \frac{1}{2} \right) -F\left( \frac{1}{4} \right)\\
& = \left[ 3 \left( \frac{1}{2} \right)^2 -2 \left( \frac{1}{2}\right)^3 \right]
- \left[ 3 \left( \frac{1}{4} \right)^2 -2 \left( \frac{1}{4} \right)^3 \right]\\
& = \left[ \frac{3}{4} -\frac{1}{4} \right] - \left[ \frac{3}{16} -\frac{1}{32} \right] \\
&= \frac{1}{2} - \frac{5}{32} = \frac{11}{32}
\end{align*}
}
\end{enumerate}
\item
\begin{enumerate}
\item -- {\em *p} --- 
{ \begin{align*}
E(X) &= \sum_i x_i \, p_i = 0 \times 0.4 + 1 \times 0.3 + 2 \times 0.2 + 4 \times 0.1\\
& = 0 + 0.3 + 0.4 + 0.4\\
&= 1.1
\end{align*}
}
\item
{
\begin{align*}
E(Y) &= \sum_i y_i \, p_i = (1+0^2)\times
 0.4 + (1+1^2) \times 0.3 + (1+2^2) \times 0.2 + (1+4^2) \times 0.1 \\
&= 1 \times 0.4 + 2 \times 0.3 +5 \times 0.2 + 17 \times 0.1\\
& = 0.4 +0.6 +1.0 + 1.7 \\
&= 3.7
\end{align*}
}
\end{enumerate}

\end{enumerate}



\Exercise {\em 5p} --
{Consider the continuous random variable, $X$, with the
probability density function
$$
f(x) = \left\{ \begin{array}{cl} \displaystyle \frac{2}{x^2}, & \mbox{if}~ 1 \leq x \leq 2   \\[2ex]
0, & \mbox{otherwise}
\end{array} \right.$$
}
\begin{enumerate}
\item
~{Find the mean value of $X$, $E(X)$.}
\item
~{Find the variance of $X$, $V(X)$.}
\item
~{If $Y = \sqrt{X}$, find the probability density function $f(y)$.
}
\end{enumerate}

\Answer
\begin{enumerate}
\item
~{
\begin{align*}
E(X) &= \int_1^2 x \, f(x) \, dx \\
&= \int_1^2 x \frac{2}{x^2} \, dx = \int_1^2 2x^{-1}\, dx = \left[ \, 2 \ln(x) \, \right]_1^2 \\
&= 2 \ln 2 - 2 \ln 1 = 2 \ln 2 \approx 1.386
\end{align*}
}
\item
~{
$$ \text{Var}(x) = E(X^2) - E(X)^2.
$$
$$
E(X^2) = \int_1^2 x^2 \frac{2}{x^2} \, dx = \int_1^2 2 \, dx = 2.
$$
Therefore,
$$ \text{Var}(X) = 2 - \left[ 2 \ln (2) \right]^2 \approx 0.0782.
$$
}
\item
~{
$$ y=\sqrt{x}, \qquad g(x) = \sqrt{x} = x^{1/2} \implies g'(x) = \frac{1}{2} x^{-1/2} = \frac{1}{2 \sqrt{x}}.$$
Hence, $\sqrt{x}$ is strictly increasing and therefore one to one. Hence $g$ is invertible. $ x =  g^{-1}(y) = y^2$.
$$ \left| \frac{d}{dy} g^{-1}(y) \right| = \frac{d}{dy} y^2 = 2y. $$
Therefore, for $1 <y < 2$,
\begin{align*}
f_Y(y) &= f_X \left( g^{-1}(y)\right) \cdot \left| \frac{d}{dy} g^{-1}(y) \right| \\
&= \frac{2}{(y^2)^2} \cdot 2y = \frac{4y}{y^4} = \frac{4}{y^3}.
\end{align*}
Thus
$$
f_Y(y) = \begin{cases} \frac{4}{y^3}, & 1 < y < \sqrt{2},\\
0, & \text{otherwise}.
\end{cases}
$$
}
\end{enumerate}

\Exercise {\em 5p} --
Let $X$ be Normal$(1,3)$, $Y$ be Normal$(-1,1)$ and $Z$ be Normal$(0,1)$ RVs that are jointly independent.
Obtain the following:
\begin{enumerate}
\item~$E(3X-3Y+4Z)$
\item~$V(2Y-2Z)$
\item~the distribution of $6-3Z+X-2Y$
\item~the probability that $6-3Z+X-2Y>0$
\item~Covariance of $X$ and $W$, where $W=X-Y$.
\end{enumerate}

\Answer
This is just a modification of the exact solution for Example 96 given below (you should find the answer for {\em this problem}! on your own):
\begin{enumerate}
\item
\[
E(3X-2Y+4Z) = 3E(X)-2E(Y)+4(Z) = (3 \times 2) + (-2 \times (-1)) + 4 \times 0 = 6 +2 +0 = 8
\]
\item
\[
V(2Y-3Z) = 2^2V(Y) + (-3)^2V(Z) = (4 \times 2) + (9 \times 1) = 8 + 9 = 17 
\]
\item
From the special property of normal RVs, the distribution of $6-2Z+X-Y$ is
\begin{align*}
&~ Normal\left( 6+(-2 \times 0)+(1 \times 2)+(-1 \times -1), ((-2)^2 \times 1) + (1^2 \times 4) + ((-1)^2 \times 2)  \right)\\
&= Normal\left( 6+0+2+1, 4 + 4 +  2  \right)\\
&= Normal(9,10)
\end{align*}
\item
Let $U=6-2Z+X-Y$ and we know $U$ is $Normal(9,10)$ RV.
\begin{align*}
P(6-2Z+X-Y>0)
&= P(U>0) = P(U-9>0-9) = P\left( \frac{U-9}{\sqrt{10}} > \frac{-9}{\sqrt{10}} \right)\\
&= P\left( Z > \frac{-9}{\sqrt{10}} \right)\\
&= P\left( Z < \frac{9}{\sqrt{10}} \right) \\
&\approxeq P(Z < 2.85) = 0.9978
\end{align*}
\item
\begin{align*}
\mathbf{Cov}(X,W)
&= E(XW)-E(X)E(W) = E(X(X-Y))-E(X)E(X-Y)\\
&= E(X^2-XY)-E(X)(E(X)-E(Y)) = E(X^2)-E(XY)-2\times(2-(-1)) \\
&= E(X^2)-E(X)E(Y)-6 = E(X^2)-(2 \times (-1))-6\\
&= (V(X)+(E(X))^2) +2-6 = (4+2^2)-4=4
\end{align*}
\end{enumerate}

\Exercise {\em 5p} --
Suppose the collection of RVs $X_1,X_2, \ldots, X_n$ model the number of errors in $n$ computer programs named $1,2,\ldots,n$, respectively.  
Suppose that the RV $X_i$ modeling the number of errors in the $i$-th program is the $\textrm{Poisson}(\lambda=3)$ for any $i=1,2,\ldots,n$.  
Further suppose that they are independently distributed.  Succinctly, we suppose that
\[
X_1,X_2,\ldots,X_n \overset{IID}{\sim} Poisson(\lambda=3) \ . 
\]
Suppose we have $n=123$ programs and want to make a probability statement about the sample mean from these $123$ samples which is the average error per program out of these $123$ programs.  
Since $E(X_i) = \lambda=3$ and $V(X_i)=\lambda=3$, we want to know how often our sample mean differs from the expectation of $3$ errors per program.
Using the CLT find the probability that the sample mean from the $123$ samples is 
\begin{enumerate}
\item ~less than $4$
\item ~less that or equal to $0$
\end{enumerate}
\Answer
See solution to Exercise 5.3 in the draft book for the course and modify it with the numbers specific to {\em this problem}!.

\Exercise {\em 5p} --
\begin{enumerate}
\item
~Show that the minimum of $k$ IID Exponential$(\lambda)$ RVs is another Exponential$(k\lambda)$ RV. Explain your argument step-by-step. 
\item ~Show that the Exponential$(\lambda)$ RV satisfies the following: \[P(X > s+t | X>s) = P(X > t).\] Explain your argument step-by-step.
\end{enumerate}
\Answer
\begin{enumerate}
\item
This was done in lectures on the last day (so only partial answer and hints on how to proceed are given). Think about what happens when $k=2$ first.
Let $Y=\min(X_1,X_2)$, then try to derive $F(Y)$ using the fact that $P(Y>y) = P(X_1>y) P(X_2>y)$.
Use the complementary DF and proceed. You will have to show your reasoning step-by-step. 
\item
This was done in lectures. It is just the proof of Proposition 26 on memorylessness of Exponential RV but with $x$ and $y$ replaced by $s$ and $t$.
\end{enumerate}

\end{ExerciseList}





\thispagestyle{scrheadings}
\ihead{}
\chead{}
\ohead{}
\ifoot{}
\cfoot{\Large \bfseries Lycka till!}
\ofoot{}
\setheadsepline{0pt}

%\newpage
\section{Standard normal distribution function table}\label{S:NormalDFTable}
%%\addcontentsline{toc}{chapter}{Standard normal distribution function table}

%\begin{table}[htp]
{\scriptsize{
For any given value $z$, its cumulative probability $\Phi(z)$ .%was generated by {\tt Excel} formula {\tt NORMSDIST}, as {\tt NORMSDIST}$(z)$.
$$
 \begin{array}{|c|c|c|c|c|c|c|c|c|c|c|c|}\hline
z       &       \Phi(z) &       z       &       \Phi(z) &       z       &       \Phi(z) &       z       &       \Phi(z) &       z       &       \Phi(z) &       z       &       \Phi(z) \\ \hline
0.01    &       0.5040  &       0.51    &       0.6950  &       1.01    &       0.8438  &       1.51    &       0.9345  &       2.01    &       0.9778  &       2.51    &       0.9940  \\
0.02    &       0.5080  &       0.52    &       0.6985  &       1.02    &       0.8461  &       1.52    &       0.9357  &       2.02    &       0.9783  &       2.52    &       0.9941  \\
0.03    &       0.5120  &       0.53    &       0.7019  &       1.03    &       0.8485  &       1.53    &       0.9370  &       2.03    &       0.9788  &       2.53    &       0.9943  \\
0.04    &       0.5160  &       0.54    &       0.7054  &       1.04    &       0.8508  &       1.54    &       0.9382  &       2.04    &       0.9793  &       2.54    &       0.9945  \\
0.05    &       0.5199  &       0.55    &       0.7088  &       1.05    &       0.8531  &       1.55    &       0.9394  &       2.05    &       0.9798  &       2.55    &       0.9946  \\
&&&&&&&&&&&\\
0.06    &       0.5239  &       0.56    &       0.7123  &       1.06    &       0.8554  &       1.56    &       0.9406  &       2.06    &       0.9803  &       2.56    &       0.9948  \\
0.07    &       0.5279  &       0.57    &       0.7157  &       1.07    &       0.8577  &       1.57    &       0.9418  &       2.07    &       0.9808  &       2.57    &       0.9949  \\
0.08    &       0.5319  &       0.58    &       0.7190  &       1.08    &       0.8599  &       1.58    &       0.9429  &       2.08    &       0.9812  &       2.58    &       0.9951  \\
0.09    &       0.5359  &       0.59    &       0.7224  &       1.09    &       0.8621  &       1.59    &       0.9441  &       2.09    &       0.9817  &       2.59    &       0.9952  \\
0.10    &       0.5398  &       0.60    &       0.7257  &       1.10    &       0.8643  &       1.60    &       0.9452  &       2.10    &       0.9821  &       2.60    &       0.9953  \\      &&&&&&&&&&&\\
0.11    &       0.5438  &       0.61    &       0.7291  &       1.11    &       0.8665  &       1.61    &       0.9463  &       2.11    &       0.9826  &       2.61    &       0.9955  \\
0.12    &       0.5478  &       0.62    &       0.7324  &       1.12    &       0.8686  &       1.62    &       0.9474  &       2.12    &       0.9830  &       2.62    &       0.9956  \\
0.13    &       0.5517  &       0.63    &       0.7357  &       1.13    &       0.8708  &       1.63    &       0.9484  &       2.13    &       0.9834  &       2.63    &       0.9957  \\
0.14    &       0.5557  &       0.64    &       0.7389  &       1.14    &       0.8729  &       1.64    &       0.9495  &       2.14    &       0.9838  &       2.64    &       0.9959  \\
0.15    &       0.5596  &       0.65    &       0.7422  &       1.15    &       0.8749  &       1.65    &       0.9505  &       2.15    &       0.9842  &       2.65    &       0.9960  \\      &&&&&&&&&&&\\
0.16    &       0.5636  &       0.66    &       0.7454  &       1.16    &       0.8770  &       1.66    &       0.9515  &       2.16    &       0.9846  &       2.66    &       0.9961  \\
0.17    &       0.5675  &       0.67    &       0.7486  &       1.17    &       0.8790  &       1.67    &       0.9525  &       2.17    &       0.9850  &       2.67    &       0.9962  \\
0.18    &       0.5714  &       0.68    &       0.7517  &       1.18    &       0.8810  &       1.68    &       0.9535  &       2.18    &       0.9854  &       2.68    &       0.9963  \\
0.19    &       0.5753  &       0.69    &       0.7549  &       1.19    &       0.8830  &       1.69    &       0.9545  &       2.19    &       0.9857  &       2.69    &       0.9964  \\
0.20    &       0.5793  &       0.70    &       0.7580  &       1.20    &       0.8849  &       1.70    &       0.9554  &       2.20    &       0.9861  &       2.70    &       0.9965  \\      &&&&&&&&&&&\\

0.21    &       0.5832  &       0.71    &       0.7611  &       1.21    &       0.8869  &       1.71    &       0.9564  &       2.21    &       0.9864  &       2.71    &       0.9966  \\
0.22    &       0.5871  &       0.72    &       0.7642  &       1.22    &       0.8888  &       1.72    &       0.9573  &       2.22    &       0.9868  &       2.72    &       0.9967  \\
0.23    &       0.5910  &       0.73    &       0.7673  &       1.23    &       0.8907  &       1.73    &       0.9582  &       2.23    &       0.9871  &       2.73    &       0.9968  \\
0.24    &       0.5948  &       0.74    &       0.7704  &       1.24    &       0.8925  &       1.74    &       0.9591  &       2.24    &       0.9875  &       2.74    &       0.9969  \\
0.25    &       0.5987  &       0.75    &       0.7734  &       1.25    &       0.8944  &       1.75    &       0.9599  &       2.25    &       0.9878  &       2.75    &       0.9970  \\      &&&&&&&&&&&\\

0.26    &       0.6026  &       0.76    &       0.7764  &       1.26    &       0.8962  &       1.76    &       0.9608  &       2.26    &       0.9881  &       2.76    &       0.9971  \\
0.27    &       0.6064  &       0.77    &       0.7794  &       1.27    &       0.8980  &       1.77    &       0.9616  &       2.27    &       0.9884  &       2.77    &       0.9972  \\
0.28    &       0.6103  &       0.78    &       0.7823  &       1.28    &       0.8997  &       1.78    &       0.9625  &       2.28    &       0.9887  &       2.78    &       0.9973  \\
0.29    &       0.6141  &       0.79    &       0.7852  &       1.29    &       0.9015  &       1.79    &       0.9633  &       2.29    &       0.9890  &       2.79    &       0.9974  \\
0.30    &       0.6179  &       0.80    &       0.7881  &       1.30    &       0.9032  &       1.80    &       0.9641  &       2.30    &       0.9893  &       2.80    &       0.9974  \\      &&&&&&&&&&&\\
0.31    &       0.6217  &       0.81    &       0.7910  &       1.31    &       0.9049  &       1.81    &       0.9649  &       2.31    &       0.9896  &       2.81    &       0.9975  \\
0.32    &       0.6255  &       0.82    &       0.7939  &       1.32    &       0.9066  &       1.82    &       0.9656  &       2.32    &       0.9898  &       2.82    &       0.9976  \\
0.33    &       0.6293  &       0.83    &       0.7967  &       1.33    &       0.9082  &       1.83    &       0.9664  &       2.33    &       0.9901  &       2.83    &       0.9977  \\
0.34    &       0.6331  &       0.84    &       0.7995  &       1.34    &       0.9099  &       1.84    &       0.9671  &       2.34    &       0.9904  &       2.84    &       0.9977  \\
0.35    &       0.6368  &       0.85    &       0.8023  &       1.35    &       0.9115  &       1.85    &       0.9678  &       2.35    &       0.9906  &       2.85    &       0.9978  \\      &&&&&&&&&&&\\

0.36    &       0.6406  &       0.86    &       0.8051  &       1.36    &       0.9131  &       1.86    &       0.9686  &       2.36    &       0.9909  &       2.86    &       0.9979  \\
0.37    &       0.6443  &       0.87    &       0.8078  &       1.37    &       0.9147  &       1.87    &       0.9693  &       2.37    &       0.9911  &       2.87    &       0.9979  \\
0.38    &       0.6480  &       0.88    &       0.8106  &       1.38    &       0.9162  &       1.88    &       0.9699  &       2.38    &       0.9913  &       2.88    &       0.9980  \\
0.39    &       0.6517  &       0.89    &       0.8133  &       1.39    &       0.9177  &       1.89    &       0.9706  &       2.39    &       0.9916  &       2.89    &       0.9981  \\
0.40    &       0.6554  &       0.90    &       0.8159  &       1.40    &       0.9192  &       1.90    &       0.9713  &       2.40    &       0.9918  &       2.90    &       0.9981  \\      &&&&&&&&&&&\\
0.41    &       0.6591  &       0.91    &       0.8186  &       1.41    &       0.9207  &       1.91    &       0.9719  &       2.41    &       0.9920  &       2.91    &       0.9982  \\
0.42    &       0.6628  &       0.92    &       0.8212  &       1.42    &       0.9222  &       1.92    &       0.9726  &       2.42    &       0.9922  &       2.92    &       0.9982  \\
0.43    &       0.6664  &       0.93    &       0.8238  &       1.43    &       0.9236  &       1.93    &       0.9732  &       2.43    &       0.9925  &       2.93    &       0.9983  \\
0.44    &       0.6700  &       0.94    &       0.8264  &       1.44    &       0.9251  &       1.94    &       0.9738  &       2.44    &       0.9927  &       2.94    &       0.9984  \\
0.45    &       0.6736  &       0.95    &       0.8289  &       1.45    &       0.9265  &       1.95    &       0.9744  &       2.45    &       0.9929  &       2.95    &       0.9984  \\      &&&&&&&&&&&\\
0.46    &       0.6772  &       0.96    &       0.8315  &       1.46    &       0.9279  &       1.96    &       0.9750  &       2.46    &       0.9931  &       2.96    &       0.9985  \\
0.47    &       0.6808  &       0.97    &       0.8340  &       1.47    &       0.9292  &       1.97    &       0.9756  &       2.47    &       0.9932  &       2.97    &       0.9985  \\
0.48    &       0.6844  &       0.98    &       0.8365  &       1.48    &       0.9306  &       1.98    &       0.9761  &       2.48    &       0.9934  &       2.98    &       0.9986  \\
0.49    &       0.6879  &       0.99    &       0.8389  &       1.49    &       0.9319  &       1.99    &       0.9767  &       2.49    &       0.9936  &       2.99    &       0.9986  \\
0.50    &       0.6915  &       1.00    &       0.8413  &       1.50    &       0.9332  &       2.00    &       0.9772  &       2.50    &       0.9938  &       3.00    &       0.9987  \\\hline
 \end{array}$$
}
}
%\end{table}
%just std normal table values

\newpage

\shipoutAnswer

\end{document}
